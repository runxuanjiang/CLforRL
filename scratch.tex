\documentclass[12pt, letterpaper]{article}
\usepackage[margin=1in]{geometry}
\usepackage{parskip}
\usepackage{amsmath}
\usepackage{amsthm}
\usepackage{amssymb}
\usepackage[]{amsfonts}
\usepackage{mathtools}
\usepackage{tikz-cd}

\usepackage[]{import}
\usepackage{pdfpages}
\usepackage{xifthen}
\usepackage{transparent}
\usepackage[]{hyperref}
\usepackage{algorithm}
\usepackage{algpseudocode}

\newcommand{\R}{\mathbb{R}}
\newcommand{\C}{\mathbb{C}}
\newcommand{\Z}{\mathbb{Z}}
\newcommand{\N}{\mathbb{N}}
\newcommand{\Q}{\mathbb{Q}}

\theoremstyle{definition}
\newtheorem*{defn}{Definition}
\newtheorem*{prop}{Proposition}
\newtheorem*{thm}{Theorem}
\newtheorem*{cor}{Corollary}
\newtheorem*{lemma}{Lemma}
\newtheorem*{rmk}{Remark}

\theoremstyle{remark}
\newtheorem*{note}{Note}
\newtheorem*{exer}{Exercise}
\newtheorem*{eg}{Example(s)}
\newtheorem*{noeg}{Non-Example(s)}

\newcommand{\incfig}[1]{
    \def\svgwidth{0.3\columnwidth}
    \import{./Figures/}{#1.pdf_tex}
}

\renewcommand\qedsymbol{$\blacksquare$}

\usepackage{subfiles}

\title{Curriculum Learning for Reinforcement Learning}
\author{Runxuan Jiang}
\date{}

\begin{document}

\maketitle

\section{Deterministic MDP}

We first consider the case of using curriculum learning on an MDP with deterministic reward and transitions.

\begin{prop}[Existence and bound for deterministic finite MDP.]
    Let \(\mathcal{M} = (\mathcal{S}, \mathcal{A}, H, P, R, \rho)\) be a deterministic MDP with sparse reward, and that \(R(s, a) = \mathcal{X}(s = g)\) where for some \(g \in A\).
    
    Then there exists a finite sequence of $H$ MDP's \(\mathcal{M}_i = (\mathcal{S}, \mathcal{A}, H, P_i, R_i, \rho_i)\) such that the following algorithm

    \begin{enumerate}
        \item Initialize \(\pi_0\) to be a random policy.
        \item For \(1 \leq i \leq H\), Follow \(\pi_{i-1}\) with an \(\epsilon_i\)-greedy policy until an optimal policy is found.
    \end{enumerate}

    will find an optimal policy for \(\mathcal{M}\) in \(O(H^2k)\) episodes in expectation, where \(k = |\mathcal{A}|\).
    
\end{prop}

\begin{proof}[Proof]
    Let \(s_0, s_1, \ldots, s_H = g\) be a trajectory that the optimal policy for \(\mathcal{M}\) takes.
    Construct \(\mathcal{M}_i\) such that \(R_i(s, a) = \mathcal{X}(s = s_i)\) and \(P_i(s_i | s_i, a) = 1 \; \forall a \in A\), otherwise \(P_i = P\) for \(1 \leq i \leq H\).

    Let \(p_i(\epsilon)\) denote the probability that the \(\epsilon\)-greedy algorithm will achieve the optimal policy for MDP \(\mathcal{M}_i\). Then we have 

    \[p_i(\epsilon) \geq f(\epsilon) := (1-\epsilon)^{i-1}(\frac{\epsilon}{k}).\]

    This comes from the fact that the optimal trajectory can be achieved by playing the greedy option \(i-1\) times, and then choosing the exploratory option and choosing the action that reaches the correct state, which has probability \(\frac{\epsilon}{k}\).

    For \(i \geq 2\), \(f\) is concave, so we can find the maximal probability by setting the derivative to 0.

    \begin{note}[]
        \color{red}
        Will need a prove on why this is concave.
    \end{note}

    \[p_i'(\epsilon) = 0\]
    \[\Rightarrow \frac{(1 - \epsilon)^{i-1}}{k} - (i-1)(1-\epsilon)^{i-2}(\frac{\epsilon}{k}) = 0\]
    \[\Rightarrow \frac{(1-\epsilon)^{i-1}}{k} = (i-1)(1-\epsilon)^{i-2}(\frac{\epsilon}{k})\]
    \[\frac{1-\epsilon}{k} = (i-1)(\frac{\epsilon}{k}) \Rightarrow \epsilon = \frac{1}{i}.\]

    By plugging in this value for \(\epsilon\) in \(f\) for \(\mathcal{M}_i\), \(p\) is bounded below by
    \[(1 - \frac{1}{i})^{i-1}\frac{1}{ki}.\]

    So the expected number of episodes to reach the optimal policy for \(\mathcal{M}_i\) is

    \[\frac{1}{(1-\frac{1}{i})^{i-1}(\frac{1}{ki})}\]
    \[= \frac{ki}{(1-\frac{1}{i})^{i-1}} = (i-1)k(\frac{i}{i-1})^i\]

    Notice that since \(i \geq 2\) and \(x \mapsto (\frac{x}{x-1})^x\) is decreasing for \(x > 0\), the above expression is bounded above by

    \[4k(i-1).\]

    So the number of episodes for the full curriculum is in expectation
    \[\sum_{j=2}^{H} 4k(j-1) \leq \frac{H}{2}(4kH)\]

    which is \(O(kH^2)\).
\end{proof}

\begin{rmk}[Tighter bound]
    Note that the above proposition only considers the case where for each \(\mathcal{M}_i\), the \(\epsilon\)-greedy algorithm is greedy for \(i-1\) steps, and then gets the correct state for the single final exploratory step. However, the curriculum learning also helps in the situation where some of the steps are exploratory and "lucky" enough to get the same action as the optimal policy, while the remaining steps are greedy.

    However, even using this the bound is still $O(kH^2)$, so this is the tightest bound we can achieve under these conditions (probably).
\end{rmk}

\begin{proof}[Proof]
    Using a similar setup as the above proof, we now consider using the probability for achieving an optimal policy:

    \[p(\epsilon) \geq f(\epsilon) := {i-1 \choose 0} (1-\epsilon)^{i-1}(\frac{\epsilon}{k}) + {i-1 \choose 1}(1-\epsilon)^{i-2}(\frac{\epsilon}{k})^2 + \ldots + {i-1 \choose i-1} (\frac{\epsilon}{k})^j\]
    \[= (\frac{\epsilon}{k})\sum_{j=0}^{i-1}{i-1 \choose j} (1-\epsilon)^{i-1-j}(\frac{\epsilon}{k})^{j}\]
    \[= (\frac{\epsilon}{k})(1-\epsilon + \frac{\epsilon}{k})^{i-1}.\]

    Note that this is concave on \((0, 1)\).
    \begin{note}[]
        \color{red}
        Proof of concavity needed.
    \end{note}

    Since it is concave, we can set the derivative to 0 to find the \(\epsilon\) that maximizes \(f\). We have
    \[f'(\epsilon) = \frac{(1-\epsilon + \frac{\epsilon}{k})^{i-1}}{k} + (\frac{\epsilon}{k}) (i-1) (1 - \epsilon + \frac{\epsilon}{k})^{i-2}(\frac{1}{k} - 1) = 0\]
    \[\Rightarrow \frac{(1 - \epsilon + \frac{\epsilon}{k})^{i-1}}{k} = (1-i)(\frac{\epsilon}{k})(\frac{1}{k} - 1)(1 - \epsilon + \frac{\epsilon}{k})^{i-2}\]
    \[\Rightarrow (1 - \epsilon + \frac{\epsilon}{k})^{i-1} = \epsilon (1-i) (\frac{1}{k} - 1) (1 - \epsilon+ \frac{\epsilon}{k})^{i-2}\]
    \[\Rightarrow 1 - \epsilon + \frac{\epsilon}{k} = \epsilon (1-i) (\frac{1}{k} - 1)\]
    \[\Rightarrow 1 - \epsilon + \frac{\epsilon}{k} = \frac{\epsilon}{k} - \epsilon + \frac{\epsilon i}{k} + i \epsilon\]
    \[\Rightarrow 1 = \epsilon(i - \frac{i}{k})\]
    \[\Rightarrow \epsilon = \frac{1}{i - \frac{i}{k}} = \frac{k}{i(k-1)}\]

    We then plug this value of \(\epsilon\) back into \(f\) to get the maximal point of \(f\):
    \[(\frac{\frac{k}{i(k-1)}}{k})(1 - \frac{k}{i(k-1)} + \frac{\frac{k}{i(k-1)}}{k})^{i-1} = \frac{1}{i(k-1)}(1 - \frac{k-1}{i(k-1)}) = \frac{1 - \frac{1}{i}}{i(k-1)}\]
    and take the reciporocal to get the expected number of episodes for \(\mathcal{M}_i\), which is
    \[\frac{i(k-1)}{1 - \frac{1}{i}} = \frac{i^2(k-1)}{i-1}\].

    Note that \(x \leq \frac{x^2}{x-1} \leq 4x\), so since \(2 \leq i \leq H\) we get that
    \[\sum_{j = 2}^H \frac{j^2 (k-1)}{j-1} = (k-1 \sum_{j=2}^H \frac{j^2}{j-1})\]
    which is \(O(kH^2)\).

    To prove this is a tight bound, we can show that the probability given by \(p(\epsilon)\) is a tight lower bound to the actual probability.
\end{proof}

\begin{prop}[Existence and bound for deterministic finite MDP]
    Let \(\mathcal{M} = (S, A, H, P, R, \rho)\) be a deterministic MDP with sparse reward, and that \(R(s, a) = \mathcal{X}(s = g)\) for some \(g \in A\).
    
    Then there exists a finite sequence of $H$ MDP's \(\mathcal{M}_i = (S, A, H, P_i, R_i, \rho_i)\) such that the following algorithm

    \begin{enumerate}
        \item Initialize \(\pi_0\) to be a random policy.
        \item For \(1 \leq i \leq H\), Follow \(\pi_{i-1}\) with an \(\epsilon_i\)-greedy policy until an optimal policy is found.
    \end{enumerate}

    will find an optimal policy with probability \(1 - \delta\) in at least \(x\) episodes.
\end{prop}

\begin{note}[]
    \color{red}
    TODO: Prove this theorem (same as above but finding the number of episodes to get an optimal policy with high probability).
\end{note}

\section{Curriculum Learning with Model-Based algorithms}

Given a model-based reinforcement learning algorithm (such as RMAX or UCB), we want to show that we can solve MDP's using the curriculum learning paradigm on the MDP's used by the algorithm to "model" the MDP to be solved. Formally, such algorithms generate a sequence of MDP's \(M_1, \ldots, M_n\) with corresponding optimal policies \(\pi^*_1, \ldots, \pi^*_n\). We want to show that by using \(\pi^*_{i-1}\) with \(\epsilon\)-greedy on \(M_i\) we can achieve optimal regret bounds for \(M_i\). Through induction, we therefore have a "curriculum" for learning the MDP.

\subsection{RMAX}

We start with an episodic nonstationery MDP \(M = (\mathcal{S}, \mathcal{A}, H, \mathbb{P}, R, \rho)\). The R-max algorithm works by encouraging exploration on "unknown" states by setting the reward for those states to a maximal value \(R_{max}\).

\subsubsection{Original R-Max Algorithm}

We first describe the RMAX algorithm for MDP's:

\begin{algorithm}[H]
\caption{R-max for MDP's}
\label{alg:rmax}
\begin{algorithmic}[1]
    \Require MDP \(M = (\mathcal{S}, \mathcal{A}, H, \mathbb{P}, R, \rho), \epsilon > 0, 0 \leq\delta \leq 1, R_{max} > 0\).
    \Procedure{R-max}{$M, \epsilon, \delta, R_{max}, K_1, K_2$}
        \State $Rewards[S][A][H]$ be a new array.
        \State $Transitions[S][A][H]$ be a new array.
        \State $Known[S][A][H]$ be a new array.
        \State $\hat{M} \gets M$ with all states replaced by absorbing states with reward \(R_{max}\) for any action.
        \State \(Done \gets False\)
        \While{not $Done$}
            \State \(Done \gets True\)
            \State \(\pi \gets \) optimal policy for \(\hat{M}\)
            \While{$\pi$ has not been run for more than $K_1$ steps}
                \State obtain trajectories by running \(\pi\) on \(M\).
                \If{a state $(s, a, h)$ has been visited more than $K_2$ times and $Known[s][a][h]$ is false}
                    \State \(Known[s][a][h] \gets True\)
                    \State set the estimated transition probabilities \(Transitions[s][a][h]\) and rewards \(Rewards[s][a][h]\) based on the average of the observed transitions and rewards for \((s, a, h)\)
                    \State replace the transition probabilities and reward in \((s, a, h)\) in \(\hat{M}\) by the estimated values \(Transitions[s][a][h]\) and \(Rewards[s][a][h]\)
                    \State \(Done \gets False\)
                    \State \textbf{break}
                \EndIf
            \EndWhile
        \EndWhile
        \State \textbf{return} The optimal policy for \(\hat{M}\)

    \EndProcedure
    
\end{algorithmic}
\end{algorithm}

\begin{note}[]
    \color{red}
    In the original R-max algorithm, the optimal policy for each \(M_j\) is not used since \(M_j\) is not know fully. Instead, the optimal policy for the estimated MDP for \(M_j\), which is based on the samples collected previously, is used.
\end{note}

\begin{rmk}[Is this curriculum learning?]
    Notice that this algorithm appears similar to curriculum learning. However there are some differences. In curriculum learning, the algorithm is given a sequence of tasks and trains an agent on the current task by using a (near)-optimal policy for the previous task. So given a sequence of MDP's \(M_1, \ldots, M_n\) you explore \(M_j\) with the aid of the policy \(\pi_{j-1}\). However, in the R-max algorithm we are exploring \(M\) directly using \(\pi_j\).
\end{rmk}

\subsubsection{R-max Algorithm with Curriculum Learning}

We first introduce some definitions. Define \(\mathcal{L} = \mathcal{P}(\mathcal{S} \times [H])\) to be the set of all possible sets of (nonstationery) states.

\begin{defn}[]
    Let \(M\) be a nonstationery finite horizon MDP as defined above and let \(L \in \mathcal{L}\). Define the MDP \underline{\(M_L\)} to be the MDP with the same states and actions as \(M\), and with the same rewards for all actions for all states \((s, h) \notin L\). For \((s, h) \in L\), the state is an absorbing state and the reward is set to \(R_{max}\) for all actions.
\end{defn}

We now introduce the curriculum learning algorithm utilizing the exploration principles of the R-max algorithm. For simplicity, we assume that the starting state for \(M\), \(s_0\) is fixed: \(M = (\mathcal{S}, \mathcal{A}, H, \mathbb{P}, R, s_0)\). We also assume that we have access to the MDP \(M_L\) for all \(L \in \mathcal{L}\).

\begin{algorithm}[H]
    \caption{Curriculum Learning with R-max}
    \label{alg:curr_rmax}
\begin{algorithmic}[1]
    \Require MDP $M = (\mathcal{S}, \mathcal{A}, H, \mathbb{P}, R, s_0)$, $\epsilon > 0$, $\delta \leq 1$, $R_{max} > 0$ and $\mathcal{M} = \{M_L: L \in \mathcal{L}\}$.
    \State Initialize $L_0 = (\mathcal{S} \times [H]) \setminus \{(s_0, 1)\}$; $M_0 = M_{L_0}$; $\bar\pi_0$ being uniformly random policy; $(\bar s_0, \bar h_0) = (s_0, 1)$.
    \State \(\mathcal{H} \gets \{\}\)
    \For{$t = 1, \dots, S \times H$}
        \For{$i = 1, \ldots, K_1$}
            \State Run \(\overline{\pi_{t-1}}\) with \(\epsilon\)-greedy on $M_{t-1}$ and add the trajectory to $\mathcal{H}$ for states in \(L \setminus L_{t-1}\).
        \EndFor
        \State Call offline policy optimization on $\mathcal{H}$ that returns $\bar \pi_{t}$, a near-optimal policy for $M_{t-1}$.
        \State \(\mathcal{H}_t \gets \{\}\)
        \For{$i = 1, \ldots, K_2$}
            \State Run \(\pi_t\) on \(M\) and add trajectory to \(\mathcal{H}_t\)
        \EndFor
        \If {$\exists (s, h) \in L : (s, h)$ is reached by at least $1$ trajectory in $\mathcal{H}_t$}
            \State \((\overline{s}_t, \overline{h}_t) \gets\) most visited state in \(L_{t-1}\)
        \Else
            \State \textbf{break}
        \EndIf
        \State $L_t = L_{t-1} / (\bar s_t, \bar h_t)$; $M_t = M_{L_t}$.
    \EndFor
    \State Output $\bar \pi_t$.
\end{algorithmic}
\end{algorithm}

We now introduce the main theoretical guarantee for the above algorithm.

\begin{thm}[]
    Given a nonstationery finite horizon MDP \(M\) with fixed initial state, an \(\epsilon > 0\), a \(0 \leq \delta \leq 1\), the maximum reward for \(M\), \(R_{max}\) and the \(\mathcal{M}\) corresponding to \(M\), then running algorithm \ref{alg:curr_rmax} with \(K_1 = \frac{-\ln(\frac{\delta}{2S})}{2(\frac{\epsilon}{SH^2 R_{max}})^2}A \frac{\ln(1 - \delta^{\frac{1}{\frac{-\ln(\frac{\delta}{2S})}{2(\frac{\epsilon}{SH^2 R_{max}})^2} A}})}{\ln(\frac{\frac{\epsilon}{HR_{max}} \sigma}{SAH})}, K_2 = \frac{-2S^2H^2 \ln(\delta)}{(\frac{\epsilon}{HR_{max}})^2}\), the algorithm will output an \(\epsilon\)-optimal policy with probability \(1 - \delta\). Furthermore, the sample complexity of the algorithm is \(SH(K_1 + K_2)\).
\end{thm}

We start off with some preliminary results about estimating the MDP, which will lead to a result justifying \(K_1\) in line 5.

\begin{thm}[]
    Let \(M\) be a MDP and let \(M'\) be another MDP such that \(||P_{M} - P_{M'}||_{\infty} < \frac{\epsilon}{SH^2R_{max}}\). Then for all policies \(\pi\),
    \[|V_{M}(\pi) - V_{M'}(\pi)| \leq \epsilon\]
\end{thm}

\begin{proof}[Proof]
    Lemma 4 of the R-max paper.
\end{proof}

\begin{lemma}[]
    Let \(M\) be an MDP. To approximate the transition probabilility for a state \(s, a, h\) in \(M\) with error less than \(\epsilon\) with probability higher than \(1 - \delta\), we need \(n \geq \frac{-\ln(\delta/2S)}{2\epsilon^2}\) samples.
\end{lemma}

\begin{proof}[Proof]
    Let \(n\) be the number of samples collected where the state \((s, a, h)\) leads to the state \(s'\). Let \(X_i\) be an indicator variable that is 1 if \((s, a, h)\) leads to \((s', h+1)\) and 0 otherwise in \(i\)th sample. Let \(p\) be the actual transition probability of transitioning from \((s, a, h)\) to \((s', h+1)\). Let \(S = \frac{X_1 + \ldots + X_{n}}{n}\). Note that \(0 \leq \frac{X_i}{n} \leq \frac{1}{n}\). Since there are \(S\) possibilities for \((s', h)\), we want the probability that \(|S - p| \geq \epsilon\) to be less than \(\frac{\delta}{S}\). By Hoeffding's inequality we have
    \[P(|S - p| \geq \epsilon) \leq 2 \exp (\frac{-2\epsilon^{2}}{n(\frac{1}{n})^2}) = 2 \exp(-2n\epsilon^2).\]
    We want
    \[2 \exp(-2n\epsilon^2) \leq \delta/S\]
    \[\Rightarrow -2n\epsilon^2 \leq \ln(\frac{\delta}{2S})\]
    \[\Rightarrow n \geq \frac{-\ln(\frac{\delta}{2S})}{2\epsilon^2}.\]
\end{proof}


\begin{lemma}[Justification for \(K_1\) (loose bound)]
    Fix \(0 \leq \delta \leq 1\). In order to collect \(n\) samples for each action played at state \((\overline{s}_t, \overline{h}_t)\) using \(\overline{\pi_{t-1}}\) in the loop in line 4 with probability \(1-\delta\) or more, we need to set \(K_1 = nA \frac{\ln(1 - \delta^{\frac{1}{nA}})}{\ln(\frac{p\epsilon}{SAH})}\).
\end{lemma}

\begin{proof}[Proof]
    The probability of \(\overline{\pi_{t-1}}\) of reaching the state \((\overline{s_t}, \overline{h_t})\) is at least \(\frac{p}{SH}\) with probability \(1 - \delta\). The probability of playing any specific action from \((\overline{s_t}, \overline{h_t})\) given that \((\overline{s_t}, \overline{h_t})\) is reached is \(\frac{\epsilon}{A}\). Thus the probability of reaching \((\overline{s_t}, \overline{h_t})\) and playing a specific action \(a\) is
    \[\frac{\epsilon p}{SAH}.\]
    We want to play each action at least \(n\) times. In this case, the probability for this is equivalent to playing a single action \(nA\) times. Let \(k\) be the proposed number of trajectories to collect. Let \(X_i, 1 \leq i \leq k\) be a Bernoulli random variable that cooresponds to a trajectory visiting \((s, h)\) and performing action \(a\) on it. Then we have
    \[P(\sum_{i=1}^{k}X_i \geq nA ) \geq \prod_{j=1}^{nA} P(\sum_{i=1}^{\frac{k}{nA}} X_{j*i} \geq 1)\]
    \[= \prod_{j=1}^{nA} (1 - P(\sum_{i=1}^{\frac{k}{nA}}X_{j*i} = 0))\]
    \[= (1 - (\frac{p\epsilon}{SAH})^{\frac{k}{nA}})^{nA} \geq \delta\]
    \[\Rightarrow 1 - (\frac{p\epsilon}{SAH})^{\frac{k}{nA}} \geq \delta^{\frac{1}{nA}}\]
    \[\Rightarrow 1 - \delta^{\frac{1}{nA}} \geq (\frac{p\epsilon}{SAH})^{\frac{k}{nA}}\]
    \[\Rightarrow \ln(1 - \delta^{\frac{1}{nA}}) \geq \frac{k}{nA} \ln(\frac{p\epsilon}{SAH})\]
    \[\Rightarrow k \geq nA \frac{\ln(1 - \delta^{\frac{1}{nA}})}{\ln(\frac{p\epsilon}{SAH})}\]
\end{proof}


Next we prove a lemma showing that if the policy in line 9 is not already an optimal policy for the MDP \(M\), then it must have a guaranteed probability of exploring \(L\).

\begin{lemma}[]
    Let \(\alpha > 0\) let \(L \subset \mathcal{S} \times [H]\). Let \(\pi_L\) be an optimal policy for \(M_L\). Let \(V_{M}(\pi)\) denote the value function on MDP \(M\) of policy \(\pi\), and let \(V^*_{M}\) be the value of any optimal policy on \(M\). Then either

    \begin{enumerate}
        \item \(V_{M} (\pi_L) > V^*_{M} - \alpha\)
        \item A state in \(L\) will be played with probability of at least \(\frac{\alpha}{H*R_{max}}\) when playing \(\pi_L\) on \(M\) (or \(M_L\)).
    \end{enumerate}
    
\end{lemma}

\begin{proof}[Proof]
    This proof follows the proof of lemma 6 in the original R-max paper.

    Suffices to show that if (2) does not hold then (1) holds. Suppose that (2) does not hold. Let \(T\) be the set of all possible trajectories on \(M\) and let \(T'\) be the set of all trajectories that pass through \(L\). Let \(P_M^{\pi}(t)\) denotes the probability that the trajectory \(t\) will be reached when playing policy \(\pi\) on the MDP \(M\).  Since (2) does not hold we have that
    \[\sum_{t \in T'} P_{M}^{\pi_L}(t) < \frac{\alpha}{H R_{max}}.\]
    We want to show that
    \[V^*_{M} - V_{M}(\pi_L) < \alpha.\]
    Note that the only differences between \(M\) and \(M_L\) is that in \(M_L\), the states in \(L\) are absorbing states and any action from this state gives a reward of \(R_{max}\), but in \(M\) the reward is replaced by the actual reward and the transition probabilities are also replaced by the actual transition probabilities from \(M\). Thus we have
    \[V^*_{M} \leq V^*_{M_L}.\]
    \[\Rightarrow V^*_{M} - V_{M}(\pi_L) \leq V^*_{M_L} - V_{M}(\pi_L)\]
    We now further decompose each of these value functions over all possible trajectories. Let \(V_{M}(t)\) denote the total reward of going through trajectory \(t\) on MDP \(M\). Then we have
    \[ |V_{M}(\pi_L) - V_{M_L}^* | = |\sum_{t \in T} P_{M}^{\pi_L}(t) V_{M}(t) - \sum_{t \in T} P_{M_L}^{\pi_L}(t) V_{M_L}(t)|\]
    \[= |\sum_{t \in T'} P_{M}^{\pi_L}(t) V_{M}(t) + \sum_{t \in T \setminus T'} P_{M}^{\pi_L}(t) V_{M}(t) - \sum_{t \in T'} P_{M_L}^{\pi_L}(t) V_{M_L}(t) - \sum_{t \in T \setminus T'} P_{M_L}^{\pi_L}(t) V_{M_L}(t)|\]
    \[\leq |\sum_{t \in T'} P_{M}^{\pi_L}(t) V_{M}(t) - \sum_{t \in T'} P_{M_L}^{\pi_L}(t) V_{M_L}(t)| + |\sum_{t \in T \setminus T'} P_{M}^{\pi_L}(t) V_{M}(t) -  \sum_{t \in T \setminus T'} P_{M_L}^{\pi_L}(t) V_{M_L}(t)|\]
    Since the rewards and transition probabilities for \(t \in T \setminus T'\) are the same for MDPs \(M\) and \(M_L\), the second expression above is equal to 0. Thus we are left with the first expression
    \[= |\sum_{t \in T'} P_{M}^{\pi_L}(t) V_{M}(t) - \sum_{t \in T'} P_{M_L}^{\pi_L}(t) V_{M_L}(t)|.\]
    Finally note that since the probabilities for all states before reaching something in \(L\) have the same transition probability for \(M\) and \(M_L\), we have that \(\sum_{t \in T'} P_{M}^{\pi_L}(t) = \sum_{t \in T'} P_{M_L}^{\pi_L}(t)\). We also have that \(\forall t, 0 \leq V_{M}(t) \leq HR_{max}\) and \(0 \leq V_{M_L}(t) \leq HR_{max}\). Thus we have
    \[|\sum_{t \in T'} P_{M}^{\pi_L}(t) V_{M}(t) - \sum_{t \in T'} P_{M_L}^{\pi_L}(t) V_{M_L}(t)| \leq \sum_{t \in T'} P_{M}^{\pi_L} H R_{max}.\]
    
    However from our assumption we have that 
    \[\sum_{t \in T'} P_{M}^{\pi_L}(t) < \frac{\alpha}{H R_{max}}.\]

    Substituting this into the previous expression we have that
    \[V^*_{M} - V_{M}(\pi_L) \leq V_{M_L}^{*} - V_M(\pi_L) < \alpha\]
    
    Which is the same statement as (1).
\end{proof}

\begin{lemma}[(Justification for \(K_2\))]
    In the loop in line 9, to guarantee that the resulting \(\overline{\pi_t}\) would have a probability of reaching \(\overline{s_t}, \overline{h_t}\) of more than \(\frac{p}{2SH}\), we need \(K_2 \geq \frac{-2S^2H^2 \ln(\delta)}{p^2}\).
\end{lemma}

\begin{proof}[Proof]
    Let \(k\) be the number of times that the for loop in line 9 is run. Let \((s, h) \in L_{i-1}\) Define the indicator random variable \(X_i\) as follows:
    \[X_i = \begin{cases}
        1 & \text{ trajectory passes through } (s, h) \\
        0 & \text{ otherwise } \\
    \end{cases}.\]

    Now define \(S_k = \sum_{i=1}^k \frac{X_i}{k}\). Then we have \(E[S_k] = P(\text{trajectory passes through } (s, h))\). By Hoeffding's inequality we have that
    \[P(E[S_k] - S_k \geq \frac{p}{2SH}) \leq \exp(\frac{-2(\frac{p}{2SH})^2}{k (\frac{1}{k})^2}) \leq \delta\]
    \[\Rightarrow \frac{-kp^2}{2S^2H^2} \leq \ln(\delta)\]
    \[\Rightarrow \frac{-kp^2}{\ln(\delta)} \leq 2S^2 H^2\]
    \[\Rightarrow -k \leq \frac{2S^2H^2 \ln(\delta)}{p^2}\]
    \[\Rightarrow k \geq \frac{-2S^2H^2 \ln(\delta)}{p^2}\]
\end{proof}

\begin{proof}[Proof (of main theorem)]
    By the above lemma, for any \(L \subset \mathcal{S} \times [H]\) we either have that the optimal policy for \(M_L\) is \(\epsilon\)-optimal for \(M\) or a state in \(L\) will  be played with probability of at least \(\frac{\epsilon}{HR_{max}}\) using the current optimal policy. So we set \(p = \frac{\epsilon}{HR_{max}}\).

    Next, let \(M_L'\) be our estimated MDP for \(M_L\). We need \(||P_{M_L} - P_{M_{L}'}||_{\infty} < \frac{\epsilon}{SH^2 R_{max}}\) which means that for each state we need (here we use \(\sigma\)-greedy)
    \[n \geq \frac{-\ln(\frac{\delta}{2S})}{2(\frac{\epsilon}{SH^2 R_{max}})^2}\]
    samples. In order to satisfy this amount for every state in \(M_L\) that is not in \(L\), and for every action played on each state, we would need
    \[K_1 = nA \frac{\ln(1 - \delta^{\frac{1}{A}})}{\ln(\frac{p\sigma}{SAH})} = \frac{-\ln(\frac{\delta}{2S})}{2(\frac{\epsilon}{SH^2 R_{max}})^2}A \frac{\ln(1 - \delta^{\frac{1}{\frac{-\ln(\frac{\delta}{2S})}{2(\frac{\epsilon}{SH^2 R_{max}})^2} A}})}{\ln(\frac{\frac{\epsilon}{HR_{max}} \sigma}{SAH})}\]
    total samples. Since we need this number of samples for each iteration, the total sample complexity of the loop in line 4 is
    \[\frac{-\ln(\frac{\delta}{2S})}{2(\frac{\epsilon}{SH^2 R_{max}})^2}SAH \frac{\ln(1 - \delta^{\frac{1}{\frac{-\ln(\frac{\delta}{2S})}{2(\frac{\epsilon}{SH^2 R_{max}})^2} A}})}{\ln(\frac{\frac{\epsilon}{HR_{max}} \sigma}{SAH})}\]

    Next we have that
    \[K_2 \geq \frac{-2S^2H^2 \ln(\delta)}{p^2} = \frac{-2S^2H^2 \ln(\delta)}{(\frac{\epsilon}{HR_{max}})^2}.\]
    Since there are \(SH\) iterations at most, the total sample complexity of the loop in line 9 is
    \[\frac{-2S^3H^3 \ln(\delta)}{(\frac{\epsilon}{HR_{max}})^2}\]

    The total sample complexity is therefore
    \[\frac{-2S^3H^3 \ln(\delta)}{(\frac{\epsilon}{HR_{max}})^2} + \frac{-\ln(\frac{\delta}{2S})}{2(\frac{\epsilon}{SH^2 R_{max}})^2}SAH \frac{\ln(1 - \delta^{\frac{1}{\frac{-\ln(\frac{\delta}{2S})}{2(\frac{\epsilon}{SH^2 R_{max}})^2} A}})}{\ln(\frac{\frac{\epsilon}{HR_{max}} \sigma}{SAH})}\]

    The algorithm is optimal because the resulting \(M_L\) would have a value function \(\epsilon\)-close to \(M\)'s value function, and the estimated \(M_{L}'\) would have a value function \(\epsilon\)-close to \(M_L\)'s value function, all with probability \(1-\delta\).
\end{proof}


\end{document}